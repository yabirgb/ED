\subsection*{Ejercicio 3}

\begin{enumerate}
\item Se trata de un algoritmo de busqueda binaria. Se parte de un
  vector ordenado y seligen dos variables que representan los extremos
  del vector. Una vez fijadas las variables \textit{sup} e
  \textit{inf} seleccionamos la posición que está en el
  \textit{'centro'} respecto a ambas. Si coincide con lo que buscamos
  hemos terminado, si es más granda cambiamos el extremo infererior a
  este valor del \textit{'centro'} y si es menor cambiamos el extremo
  superior.

\item La complejidad teórica es $log_2(n)$ ya que fuera del bucle
  \textit{while} todo tiene complejidad $O(1)$ y dentro, en cada
  iteración, reducimos la condición en un orden de dos. Por tanto el
  algoritmo tiene la mayor de las complejidades, $log_2(n)$.
\end{enumerate}

