\documentclass[a4paper,11pt]{article}
\usepackage[left=3cm,right=3cm,top=2cm,bottom=2cm]{geometry} % page settings
\usepackage{amsmath} % provides many mathematical environments & tools
\usepackage[spanish]{babel}

\selectlanguage{spanish}
\usepackage[utf8]{inputenc}
\setlength{\parindent}{0mm}

\usepackage{graphicx}
\usepackage{float}

\usepackage{listings}

\newenvironment{ejercicio}[1]{\textbf{#1} \vspace*{0mm}}{\vspace*{5mm}}
\setlength{\parindent}{10pt} 

\begin{document}

\title{Práctica de eficiencia}
\author{Yábir García Benchakhtir}
\date{\today}
\maketitle

\subsection*{Especificaciones técnicas}

\begin{flushleft}
  Los ejercicios de esta práctica se han realizado en un ordenador de
  sobremesa con las siguientes características:
\end{flushleft}

\begin{itemize}
\item CPU: Intel® Pentium(R) CPU G3258 @ 3.20GHz x 2
\item RAM: 7,7 GiB DDR3
\item SO: Ubuntu Linux 16.04.4 LTS 64-bits
\end{itemize}

\begin{flushleft}
  Como compilador se ha usado \textit{g++} en la version \texttt{g++
    (Ubuntu 5.4.0-6ubuntu1~16.04.4) 5.4.0 20160609}

  
  \textit{Los archivos .fit y los programas cpp utilizados se adjuntan
    en un comprimido clasificados por ejercicio.}

\end{flushleft}

\subsection*{Ejercicio 4}

Para crear un vector ordenado he usado la siguiente modificación:

\begin{verbatim}
  v[0] = 0;
  for (int i=1; i<tam; i++){  // Recorrer vector
    v[i] = v[i-1] + rand() % 50;    // Generar aleatorio [0,vmax[
  }
\end{verbatim}

y para crear un vector decreciente:

\begin{verbatim}
  v[0] = vmax;
  for (int i=1; i<tam; i++){  // Recorrer vector
    v[i] = v[i-1] - rand() % 100;    // Generar aleatorio [0,vmax[
  }
\end{verbatim}

El resultado ha sido el siguiente:

\begin{figure}[H]
  \caption{Comparación de modificaciones en el algoritmo bubble sort}
  \centering
    \includegraphics[width=0.8\textwidth]{final.png}
\end{figure}
\newpage
\subsection*{Ejercicio 4}

Para crear un vector ordenado he usado la siguiente modificación:

\begin{verbatim}
  v[0] = 0;
  for (int i=1; i<tam; i++){  // Recorrer vector
    v[i] = v[i-1] + rand() % 50;    // Generar aleatorio [0,vmax[
  }
\end{verbatim}

y para crear un vector decreciente:

\begin{verbatim}
  v[0] = vmax;
  for (int i=1; i<tam; i++){  // Recorrer vector
    v[i] = v[i-1] - rand() % 100;    // Generar aleatorio [0,vmax[
  }
\end{verbatim}

El resultado ha sido el siguiente:

\begin{figure}[H]
  \caption{Comparación de modificaciones en el algoritmo bubble sort}
  \centering
    \includegraphics[width=0.8\textwidth]{final.png}
\end{figure}
\subsection*{Ejercicio 4}

Para crear un vector ordenado he usado la siguiente modificación:

\begin{verbatim}
  v[0] = 0;
  for (int i=1; i<tam; i++){  // Recorrer vector
    v[i] = v[i-1] + rand() % 50;    // Generar aleatorio [0,vmax[
  }
\end{verbatim}

y para crear un vector decreciente:

\begin{verbatim}
  v[0] = vmax;
  for (int i=1; i<tam; i++){  // Recorrer vector
    v[i] = v[i-1] - rand() % 100;    // Generar aleatorio [0,vmax[
  }
\end{verbatim}

El resultado ha sido el siguiente:

\begin{figure}[H]
  \caption{Comparación de modificaciones en el algoritmo bubble sort}
  \centering
    \includegraphics[width=0.8\textwidth]{final.png}
\end{figure}
\subsection*{Ejercicio 4}

Para crear un vector ordenado he usado la siguiente modificación:

\begin{verbatim}
  v[0] = 0;
  for (int i=1; i<tam; i++){  // Recorrer vector
    v[i] = v[i-1] + rand() % 50;    // Generar aleatorio [0,vmax[
  }
\end{verbatim}

y para crear un vector decreciente:

\begin{verbatim}
  v[0] = vmax;
  for (int i=1; i<tam; i++){  // Recorrer vector
    v[i] = v[i-1] - rand() % 100;    // Generar aleatorio [0,vmax[
  }
\end{verbatim}

El resultado ha sido el siguiente:

\begin{figure}[H]
  \caption{Comparación de modificaciones en el algoritmo bubble sort}
  \centering
    \includegraphics[width=0.8\textwidth]{final.png}
\end{figure}
\subsection*{Ejercicio 4}

Para crear un vector ordenado he usado la siguiente modificación:

\begin{verbatim}
  v[0] = 0;
  for (int i=1; i<tam; i++){  // Recorrer vector
    v[i] = v[i-1] + rand() % 50;    // Generar aleatorio [0,vmax[
  }
\end{verbatim}

y para crear un vector decreciente:

\begin{verbatim}
  v[0] = vmax;
  for (int i=1; i<tam; i++){  // Recorrer vector
    v[i] = v[i-1] - rand() % 100;    // Generar aleatorio [0,vmax[
  }
\end{verbatim}

El resultado ha sido el siguiente:

\begin{figure}[H]
  \caption{Comparación de modificaciones en el algoritmo bubble sort}
  \centering
    \includegraphics[width=0.8\textwidth]{final.png}
\end{figure}
\subsection*{Ejercicio 4}

Para crear un vector ordenado he usado la siguiente modificación:

\begin{verbatim}
  v[0] = 0;
  for (int i=1; i<tam; i++){  // Recorrer vector
    v[i] = v[i-1] + rand() % 50;    // Generar aleatorio [0,vmax[
  }
\end{verbatim}

y para crear un vector decreciente:

\begin{verbatim}
  v[0] = vmax;
  for (int i=1; i<tam; i++){  // Recorrer vector
    v[i] = v[i-1] - rand() % 100;    // Generar aleatorio [0,vmax[
  }
\end{verbatim}

El resultado ha sido el siguiente:

\begin{figure}[H]
  \caption{Comparación de modificaciones en el algoritmo bubble sort}
  \centering
    \includegraphics[width=0.8\textwidth]{final.png}
\end{figure}
\subsection*{Ejercicio 4}

Para crear un vector ordenado he usado la siguiente modificación:

\begin{verbatim}
  v[0] = 0;
  for (int i=1; i<tam; i++){  // Recorrer vector
    v[i] = v[i-1] + rand() % 50;    // Generar aleatorio [0,vmax[
  }
\end{verbatim}

y para crear un vector decreciente:

\begin{verbatim}
  v[0] = vmax;
  for (int i=1; i<tam; i++){  // Recorrer vector
    v[i] = v[i-1] - rand() % 100;    // Generar aleatorio [0,vmax[
  }
\end{verbatim}

El resultado ha sido el siguiente:

\begin{figure}[H]
  \caption{Comparación de modificaciones en el algoritmo bubble sort}
  \centering
    \includegraphics[width=0.8\textwidth]{final.png}
\end{figure}
\subsection*{Ejercicio 4}

Para crear un vector ordenado he usado la siguiente modificación:

\begin{verbatim}
  v[0] = 0;
  for (int i=1; i<tam; i++){  // Recorrer vector
    v[i] = v[i-1] + rand() % 50;    // Generar aleatorio [0,vmax[
  }
\end{verbatim}

y para crear un vector decreciente:

\begin{verbatim}
  v[0] = vmax;
  for (int i=1; i<tam; i++){  // Recorrer vector
    v[i] = v[i-1] - rand() % 100;    // Generar aleatorio [0,vmax[
  }
\end{verbatim}

El resultado ha sido el siguiente:

\begin{figure}[H]
  \caption{Comparación de modificaciones en el algoritmo bubble sort}
  \centering
    \includegraphics[width=0.8\textwidth]{final.png}
\end{figure}

\end{document}