\documentclass{article}
\usepackage[left=3cm,right=3cm,top=2cm,bottom=2cm]{geometry} % page settings
\usepackage{amsmath} % provides many mathematical environments & tools
\usepackage{amssymb}
\usepackage{amsfonts}
\usepackage[spanish]{babel}



\usepackage{multirow}

\usepackage[]{algorithm2e}

\usepackage[utf8]{inputenc}
\setlength{\parindent}{0mm}

\usepackage[parfill]{parskip}

% Para el código
\usepackage{listings}
\usepackage{xcolor}
\definecolor{gray}{rgb}{0.5,0.5,0.5}
\newcommand{\n}[1]{{\color{gray}#1}}
\lstset{numbers=left,numberstyle=\small\color{gray}}

% Entorno para estilo de ejercicios
\newenvironment{ejercicio}[1]{\textbf{#1} \vspace*{5mm}}{\vspace*{5mm}}
\setlength{\parindent}{10pt} 

\begin{document}

\title{Estructura de datos - Reto 2}
\author{Yábir García Benchakhtir}
\date{\today}
\maketitle

\section{Definición del problema}

Consideramos los conjuntos $I_{10} = \{1,2,3,4,5,6,7,8,9,10\}$ y
$B = \{25,50,75,100\}$ y consideramos elementos
$C \in I_{10}^{5} \times B$.

Se nos plantean los siguientes problemas:

\begin{itemize}
\item Dado un elemento $C$ y un número $n$ tal que $1 \le n \le 999$
  comprobar si $n \in L(C)$ donde notamos con $L(C)$ los resultados
  posibles de operar con los elementos de $C$.

\item Determinar el conjunto de operaciones necesarias para obtener
  $n$ a partir de $C$.

\item Determinar si un vector $C$ es una \textit{cadena mágica}, es
  decir, un vector tal que $L(C)$ contiene todos los elementos en el
  rango $1 \le n \le 999$.
\end{itemize}

\section{Acotación del problema}

Podemos \textit{abstraer} el problema y no limitarlo unicamente al
conjunto de operaciones $\{+,-,*,/\}$ usuales. Podemos usar un
conjunto \textit{finito} de operaciones que notaremos como
$\{p_1, p_2, ..., p_n\}$ (Podemos conseguir que el programa funcione con operaciones
de ariedad $\ge 2 $).

Estás operaciones pueden tener restricciones en su definición tales
como ser conmutativas o no serlo, no poder tomar ciertos valores, etc.

Así podemos crear \textit{n-uplas} del tipo $O = <p_i, r, l, rep>$
donde:

\begin{itemize}
\item $p_i$ es la operación que se va a usar.
\item $r$ es un valor de tipo \textbf{Bool} indicando si tiene o no
  restricciones.
\item $l$ es una función que evalua los operandos de $p_i$
\item $rep$ es la representación que se va a usar para la operación. 
\end{itemize}

Por ejemplo si consideramos la operación $"-"$ usual podemos imponer la
restricción de que para $a,b \in C$ tienen que verficar $a-b \ge 1$

Así consideramos la estructura $<N(C), O>$ donde $N(C)$ es el vector
de números hecho nodos como base para el procedimiento y crearemos
para cada iteración del algorimo
$<n, {n_1, ..., n_k} , p_i, gen>$, lo que hemos llamaremos
\textit{nodo} y que contiene:

\begin{itemize}
\item $n$ - Valor del nodo
\item ${n_1, ..., n_k}$ - Conjunto de nodos que originan el nodo.
\item $p_i$ - Operación que origina el nodo.
\item $gen$ - Número de generación.
\end{itemize}

\section{Descripción del algoritmo}

En esta implementación vamos a \textit{recrear} un proceso
\textit{evolutivo}.  Partimos de un vector
$C \in I_{10}^{5} \times B$. La estrategia consistirá en formar
parejas de elementos de $<a,b> \in P(N(C))$ y reducir en cada paso en
1 la dimensión del vector $N(C)$ hasta que obtengamos un único nodo.

\begin{algorithm}[H]
 \KwData{$<N(C),O>$}
 \KwResult{A finite array of nodes}
 initialization\;
 \While{not at end of this document}{
  read current\;
  \eIf{understand}{
   go to next section\;
   current section becomes this one\;
   }{
   go back to the beginning of current section\;
  }
 }
 \caption{Definición del algoritmo}
\end{algorithm}

\section{Problemas de esta implementación y alternaticas}

La implementación realizada presenta como principal problema el coste
de memoria ya que tiene que almacenar los datos resultantes de cada
iteración del algoritmo.

Se ha elejido esta implementación por varios motivos:

\begin{itemize}
\item Es sencillo comprobar si una cadena es mágica o no.
\item Para un mismo valor $n$ ofrece todas las posibles manera de
  construirlo a partir de $C$.
\end{itemize}

Como posibles solucines podemos intentar no recorrer el \textit{arbol}
por capas y recorrerlo en su lugar por ramas de modo que se resuelve
el problema de la memoria.

En tests \textit{rudimentarios} realizados en Ubuntu consume cerca de
\textit{150MB} de espacio RAM.

\end{document}
