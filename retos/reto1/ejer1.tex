
\begin{ejercicio}
  {1. Usando la notación O , determinar la eficiencia de los siguientes segmentos de
    código:}

  \begin{lstlisting}[
      language=C++, directivestyle={\color{black}},
      emph={int,char,double,float,unsigned}]
      int n, j; int i=1; int x=0;
      do{
        j = 1;
        while(j <= n){
          j=j*2;
          x++;
        }
        i++;
      }while(i <= n)
    \end{lstlisting}

    \begin{flushleft}
      En la primera línea tenemos operaciones de asignación a
      variables con complejidad $O(1)$. Encontramos un bucle
      \textit{do-while} (líneas 2-9) que contiene:
    \end{flushleft}


    \begin{itemize}
    \item Una asignación de variable, $O(1)$.
    \item Un bucle \textit{while} que se repite $log_2n$ veces con
      operaciones $O(1)$. La complejidad de esta estructura es pues
      $log_2(n)$.
    \item Una operación de incremento que contamos como $O(1)$.
    \end{itemize}

    \begin{flushleft}
      La complejidad de esta última estructura es por tanto $nlog_2n$
      ya que el bucle se repite $n$ veces y la mayor complejidad en
      terminos de Big O es $log_2n$.  Así la complejidad que obtenemos
      para el código completo es:
    \end{flushleft}

    \[
      O(\max\{1, nlog_2n\} = O(nlog_2n)
    \]

     \begin{lstlisting}[
      language=C++, directivestyle={\color{black}},
      emph={int,char,double,float,unsigned}]
      int n, j; int i=2; int x=0;
      do{
        j = 1;
        while(j <= i){
          j=j*2;
          x++;
        }
        i++;
      }while(i <= n)
    \end{lstlisting}

    \begin{flushleft}
      En el segundo caso estamos repitiendo todo el contenido del
      \textit{do-while} $n$ veces y dentro de este bucle tenemos una
      sentencia \textit{while} que depende de una variable $j$ que aumenta
      de dos en dos. Dentro de este solo tenemos sentencios de complejidad $O(1)$
      luego la expresión de la complejidad es del tipo:
    \end{flushleft}


  \[
    \sum\limits_{i=1}^{n}{log_2(i)} = log_2{\prod\limits_{i=1}^{n}i} = log_2(n!)
  \]


\end{ejercicio}
