\documentclass{article}
\usepackage[left=3cm,right=3cm,top=2cm,bottom=2cm]{geometry} % page settings
\usepackage{amsmath} % provides many mathematical environments & tools
\usepackage{amssymb}
\usepackage{amsfonts}
\usepackage[spanish]{babel}



\usepackage{multirow}

\usepackage{algorithm}
\usepackage{algpseudocode}
\usepackage{pifont}

\usepackage[utf8]{inputenc}
\setlength{\parindent}{0mm}

\usepackage[parfill]{parskip}

% Para el código
\usepackage{listings}
\usepackage{xcolor}
\definecolor{gray}{rgb}{0.5,0.5,0.5}
\newcommand{\n}[1]{{\color{gray}#1}}
\lstset{numbers=left,numberstyle=\small\color{gray}}

% Entorno para estilo de ejercicios
\newenvironment{ejercicio}[1]{\textbf{#1} \vspace*{5mm}}{\vspace*{5mm}}
\setlength{\parindent}{10pt} 

\begin{document}

\title{Estructura de datos - Reto 5}
\author{Yábir García Benchakhtir}
\date{\today}
\maketitle


\section{Problema}

Se nos da una función \textit{hash} definida como: $$ h(k) = k \% M $$
y un sistema para resolver colisiones siguiendo un esquema de hashing doble mediante la función:
$$ h_i(k) = [h(k) + d_i] \% M $$ con i un número natural mayor que 1 y $d_i$ definido como:

\begin{align*}
  d_0 &= 0 \\
  d_i &= (a*d_{i-1}+c) \% M
\end{align*}

A partir de estas condiciones se nos pide que fijado un número $M \in \mathbb{Z}^+$
primo discutamos cuales son los pares $<a,c>$ que provocan que usados
en la definición anterior la función ocupe todas las posiciones
disponibles antes de repetir posiciones, es decir, nos garanticemos
que tenemos acceso a todas las posiciones de la tabla.

\section{Encontrar los pares}

Por definición de $d_i$ los valores de $a$ y $c$ que tenemos que
comprobar abarcan desde 0 a $M-1$ ya que estamos realizando una
operación de módulo. Por ello si definimos $I_N = \{0 .. N\}$ tenemos que buscar
$$\{<a,c> \in I_{M-1}\times I_{M-1}\}$$ que completan todos los huecos accesibles de nuestra
tabla \textit{hash}.

Para ello vamos a introducir $M$ veces el mismo valor en la tabla de
modo que provoquemos el máximo de colisiones posibles y podamos
comprobar si todos los huecos de nuestra tabla se rellenan.

Utilizaremos en esta implementación un vector con $M$ posiciones y
emplearemos $-$ como simbolo para marcar que una casilla está disponible.

En cada iteración comprobamos si es posible insertarlo en la posición
que le asigna $h$ y si no es posible llamamos a $h_i$ hasta que
determinemos que tiene que existir un ciclo.

Se adjunta un programa en python que realiza estos calculos usando el
criterio anterior.

La implementación tiene complejidad $O(n^3)$ luego hacer calculos para
números relativamente pequeños a se vuelve una tarea compleja.

\section{Resultados obtenidos}

Si ejecutamos el programa para números primos en $\mathbb{Z}^+$
obtenemos que las parejas que completan todas las posiciones son los
pares de la forma $<1, n>$ con $0 < n < M-1$.

Este resultado se explica si estudiamos como aumenta el valor de $d_i$
en función de su indice.

Obtenesmos una expresión de la forma:

$$ \left[ \sum_{n=0}^{i-1} a^nc \right] \% M $$

De esta forma cuando encontramos una colisión incrementamos un grado
este coeficiente.

\section{¿Por qué funciona este método?}

Mediante la función $h(k)$ insertamos en la posición que le
correspondería y mediante $h_i(k)$ lo que hacemos es incrementar la
posición que le corresponde sumandole un incremento positivo que
depende de la expresión expuesta en el apartado anterior.

Cuando usamos los pares de la forma $<1,n>$ realizamos un incremento
de una unidad en cada iteración que hacemos, es decir, si $h(x) = 0$

\begin{align*}
  h_1(x) &= (0 + c) \% M \\
  h_2(x) &= (0 + 2\cdot c) \% M \\
  h_3(x) &= (0 + 3\cdot c) \% M
\end{align*}

y esto nos recorre de forma lineal todas las posiciones de la lista
con lo que nos garantizamos que accedemos a todas las posiciones
disponibles.

\end{document}
