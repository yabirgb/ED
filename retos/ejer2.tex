
\begin{ejercicio}
  {2. Para cada función $f(n)$ y cada tiempo $t$ de la tabla siguiente, determinar el mayor tamaño de un problema que puede ser resuelto en un tiempo t (suponiendo que el algoritmo para resolver el problema tarda $f(n)$ microsegundos, es decir, $f(n) \time 10 ^{-6} sg$ .}
\end{ejercicio}


\begin{flushleft}
  Para ello vamos a obtener una expresión sencilla de la función que
  la relaciones con el tiempo que se nos pide $t = f(n)\cdot 10^6$.
\end{flushleft}

\begin{itemize}
\item $f(n) = log_{2}$
  \begin{flushleft}
    Tenemos que despejar el valor de n en la expresión: $10^{-6}log_2n = t$
  \end{flushleft}
  \[
    10^{-6}log_2n = t \implies log_2n = t\cdot 10^6 \implies 2^{t\cdot 10^6} = n
  \]

\item $f(n) = nlog_2n$
  \[
    nlog_2n = t\cdot 10^6
  \]

  \begin{flushleft}
    En este caso no existe expresión analitica que podamos simplificar para resolver esta ecuación.
  \end{flushleft}

\item $f(n) = n$
  \[
    n = t \cdot 10^6
  \]

\item $f(n) = n^3$
  \[
    n = \sqrt[3]{t\cdot 10^6} = 100 \sqrt[3]{t}
  \]

\item $f(n) = 2^n$

  \[
    2^n = t \cdot 10^6 \implies n = log_2(t\cdot 10^6)
  \]

\item $f(n) = n!$

  \[
    n! = t\cdot 10^6
  \]
  
\end{itemize}

\begin{flushleft}
  Usando las expresiones anteriores despejamos los valores de n para los tiempos dados:
\end{flushleft}


\begin{table}[!htbp]
\centering
\label{Tabla}
\begin{tabular}{|c|c|c|c|c|c|}
\hline
 \multirow{2}{*}{$f(n)$}& \multicolumn{5}{l|}{\hfil$t$} \\ \cline{2-6} 
 &  1 segundo  & 1 hora   & 1 semana   &  1 año  & 1000 años  \\ \hline
 $log_2n$ &  $9.9 \cdot 10^{301029}$  &  $2^{3600\cdot 10^6}$  &  $2^{6.04\cdot 10^{12}}$  &  $2^{3.15\cdot 10^{13}}$  &  $2^{3.15\cdot 10^{16}}$ \\ \hline
 $n$&  $10^6$  &  $3.600\cdot 10^9 $  &  $6.048 \cdot 10^{11}$  &  $3.1536 \cdot 10^{13}$  & $3.1536 \cdot 10^{16}$  \\ \hline
 $nlog_2n$&  $62746$  &  $1.33378\cdot 10^8$  &  $1.77631\cdot 10^{10}$  &  $7.97634\cdot 10^{11}$  &  $6.41137\cdot 10^{14}$ \\ \hline
 $n^3$&  $100$  &  $1532$  &  $8456$  &  $31593$  & $315938$  \\ \hline
 $2^n$&  $19$  &  $31$  & $39$   &  $44$  & $54$  \\ \hline
 $n!$&  $9$  &  $12$  &  $14$  &  $16$  &  $18$ \\ \hline
\end{tabular}
\end{table}